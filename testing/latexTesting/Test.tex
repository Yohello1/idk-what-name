\documentclass[12pft, english]{article}
\usepackage{apacite}
\usepackage[]{multicol}
\usepackage[a4paper, bindingoffset=0.2in, left=0.25in, right=0.25in, top=0.5in, bottom=0.5in]{geometry}
\title{Investigating whether the inputs of a Multiplicative Congruential Generator can be determined using the outputs}
\author{Yohwllo}
\date{March 10th 2024}


\begin{document}
\maketitle

\tableofcontents
\pagebreak

\begin{multicols}{2}
  \section{Introduction}
  Ever since I was a young child the concept of random has amazed me. For randomness could at times not seem random, for example, number generators more often than not are far from random, instead they are psuedo random number generators, which are not truely random. These pesudo random number genrators are not truely random, but rather just a mathematical equation which is being done itteratively, and if the developer was forward thinking, it used a different seed every time too. \\
  Now among random number generators there is a particularly common one called the Multiplicative Congruential Generator

\end{multicols}

Some asdclaisdm \cite{hull1962random}.

\bibliographystyle{apacite}
\bibliography{References}
\end{document}
