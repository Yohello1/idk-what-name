\documentclass{article}
\usepackage[]{amsfonts}
\usepackage[]{amsmath}

\title{iradiated maze}
\author{Siracha}
\date{March}

\begin{document}

After putting all the scrap metal infront of the spacecraft, you and john get a bit bored, and decide to pull out some games.
One of these games is a game about mazes, and free spaces.
But all the boards were written down in a weird way, so we have to decode them, and then solve/play them.

To earn points in the game, you have to see how many free squares are accessible from the point `(0,0)`. Which is the top left corner of the game board. A free square is denoted by a '0' at that point, and a occupied square is denoted by a '1' in that square, if `(0,0)` is occupied, then no squares are reachable.

But before we can play the game, we'll need to decode the weird boards. Every board is made up of 1's, and 0's, and is square. This would mean that a 3x3 board would look like:
011
101
101
But it was stored like this:
1F2T;
1T1F1T;
1T1F1T;

To decode it, we need to look at the number, and character after it. The number is the amount of times that the same value appears in sequential order. In this case, `F` means `0`, and `T` means `1`. Finally the semi-colon denotes the end of a line.

You've gotta take the board, and figure out the score of that board, and return it as an integer.
You'll have to make a function called `solve_board`, which takes a string(the board) and returns an integer(the score)

The board will be between 1, and 1000 values wide.


\end{document}
