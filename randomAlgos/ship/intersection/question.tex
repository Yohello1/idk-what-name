\documentclass{article}
\usepackage{graphicx} % Required for inserting images
\usepackage[]{amsfonts}
\usepackage[]{amsmath}


\title{sphehes intersection}
\author{Siracha}
\date{March }

\begin{document}

\maketitle

\section{Introduction}


\title{Math IA}
\author{}
\date{March 2024}


\maketitle

\section{Problem}

After John saved the Johnville from imminent destruction from flooding, the townspeople invited him over to celebrate. John of course went to this celebration, where they ate, drank, and talked a lot. John also brought his trusty C64 along with him, he's kind of grown to love it. During the party the celebration the mayor of Johnville told John, that they were gonna send a rocket to space and wanted John to man the space craft. You're going on this trip too.

John, said yes without a second thought... Until he woke up, and sobered up and relised, *I am not an astronaut*. But it was too late.

A few months pass by and the day of depature arrives. You and John get into the space craft, preparing for take off.
T-3, T-2, T-1, are the things you hear before the engines roar to life and your body gets pushed against the seat.
8 minutes later, you're in space. But engines haven't shut off? You check the navigational computer to see where you're supposed to be going, and you find out. YOU'RE HEADED RIGHT TOWARDS A LOT OF PLANETS AND METORS!!!

The space ship cant withstand going through all those metors and planets. So John devises a brilliant plan to save you guys. Putting stuff in front of the space ship to sheild it from the on coming stuff.

But you've gotta figure out how much stuff to put infront of the space craft.
Now thankfully the navigational computer has the locations (center) of all the planets, and their radiuses. They're also all made of the same material, and of the same density. This saves us a lot of time, and energy. But now we have another problem, how are we gonna calculate it? We don't have our phones, or a computer... It's at that moment, John takes out his C64, which he hid in his space suit.

Using the C64 we will need to calcuate the amount of scrap metal that we'll need to put infront of the ship. Doing some rough calculations we figure out that we'll need 1kg of scrap metal per meter of stuff we'll be going through.

Now your job is to take the information given, and make a function called `scrap metal amt calc`, which will take the locations of the planets, and it's radius as two seperate lists. Then return the amount of scrap metal needed as an integer (round up).


\end{document}
